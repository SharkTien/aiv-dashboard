\documentclass[12pt,a4paper]{article}
\usepackage[utf8]{inputenc}
\usepackage[vietnamese]{babel}
\usepackage{amsmath}
\usepackage{amsfonts}
\usepackage{amssymb}
\usepackage{graphicx}
\usepackage{listings}
\usepackage{xcolor}
\usepackage{geometry}
\usepackage{hyperref}
\usepackage{fancyhdr}
\usepackage{titlesec}

\geometry{margin=2.5cm}
\pagestyle{fancy}
\fancyhf{}
\fancyhead[L]{Báo cáo thực tập}
\fancyhead[R]{AIESEC Dashboard CI/CD}
\fancyfoot[C]{\thepage}

% Cấu hình cho code blocks
\lstset{
    language=bash,
    basicstyle=\ttfamily\small,
    keywordstyle=\color{blue},
    commentstyle=\color{gray},
    stringstyle=\color{red},
    numbers=left,
    numberstyle=\tiny,
    stepnumber=1,
    numbersep=5pt,
    backgroundcolor=\color{gray!10},
    showspaces=false,
    showstringspaces=false,
    showtabs=false,
    frame=single,
    tabsize=2,
    captionpos=b,
    breaklines=true,
    breakatwhitespace=false,
    escapeinside={\%*}{*)}
}

\title{\textbf{CHƯƠNG 4: TRIỂN KHAI HỆ THỐNG CI/CD}}
\author{}
\date{}

\begin{document}

\maketitle

\section{Quy trình lắp đặt hệ điều hành và deploy lên VPS}

Mục tiêu của phần này là mô tả đầy đủ các bước từ một VPS ``trắng'' (chỉ có SSH) tới khi ứng dụng Next.js chạy ổn định kèm cơ sở dữ liệu MySQL và giao diện quản trị Adminer. Các bước áp dụng trên Ubuntu 22.04 LTS.

\subsection{Khởi tạo và bảo mật cơ bản}

\begin{lstlisting}[caption=Cập nhật hệ thống và tạo user quản trị]
# Cập nhật OS
sudo apt update && sudo apt upgrade -y

# Tạo user mới (ví dụ: deploy)
sudo adduser deploy
sudo usermod -aG sudo deploy

# Kích hoạt firewall cơ bản
sudo apt install ufw -y
sudo ufw allow OpenSSH
sudo ufw allow 80
sudo ufw allow 443
sudo ufw enable
sudo ufw status
\end{lstlisting}

\subsection{Cấu hình SSH key-based login}

\begin{enumerate}
  \item Tạo SSH key trên máy Windows (PowerShell):
\begin{lstlisting}
ssh-keygen -t ed25519 -C "email@domain"  # sinh C:\\Users\\YOU\\.ssh\\id_ed25519(.pub)
\end{lstlisting}
  \item Dán public key vào VPS cho user \texttt{deploy}:
\begin{lstlisting}
sudo -u deploy mkdir -p /home/deploy/.ssh
sudo nano /home/deploy/.ssh/authorized_keys  # dán nội dung id_ed25519.pub
sudo chown -R deploy:deploy /home/deploy/.ssh
sudo chmod 700 /home/deploy/.ssh
sudo chmod 600 /home/deploy/.ssh/authorized_keys
sudo chmod 755 /home /home/deploy
\end{lstlisting}
  \item Chỉnh \texttt{/etc/ssh/sshd\_config} (tùy chọn, tăng bảo mật):
\begin{lstlisting}
PubkeyAuthentication yes
PasswordAuthentication no
PermitRootLogin prohibit-password
AuthorizedKeysFile .ssh/authorized_keys
\end{lstlisting}
  \item Khởi động lại SSH:
\begin{lstlisting}
sudo systemctl reload ssh
\end{lstlisting}
\end{enumerate}

\subsection{Cài đặt Node.js, PM2, Git}

\begin{lstlisting}[caption=Cài Node.js LTS, PM2, Git]
curl -fsSL https://deb.nodesource.com/setup_lts.x | sudo -E bash -
sudo apt install -y nodejs git

sudo npm i -g pm2
pm2 --version
git --version
node -v && npm -v
\end{lstlisting}

\subsection{Triển khai ứng dụng Next.js với PM2}

\begin{lstlisting}[caption=Clone mã nguồn và chạy bằng PM2]
sudo mkdir -p /root/apps/aiv-dashboard
sudo chown -R $USER:$USER /root/apps/aiv-dashboard || true
cd /root/apps/aiv-dashboard

# Clone (hoặc scp) project
git clone https://github.com/your-org/aiv-dashboard.git .

# Cài dependencies và build
npm ci
npm run build

# Tạo ecosystem PM2
cat > ecosystem.config.js <<'EOF'
module.exports = {
  apps: [{
    name: 'aiv-dashboard',
    script: 'npm',
    args: 'run start -- -p 3000 -H 0.0.0.0',
    cwd: '/root/apps/aiv-dashboard',
    env_file: '.env'
  }]
}
EOF

# Tạo file .env theo ứng dụng
cat > .env <<'EOF'
DATABASE_HOST=127.0.0.1
DATABASE_PORT=3306
DATABASE_USER=root
DATABASE_PASSWORD=yourStrongPass
DATABASE_NAME=aivdb
DATABASE_SSL=false
EOF

# Khởi chạy bằng PM2
pm2 start ecosystem.config.js
pm2 save
pm2 startup  # Thiết lập tự khởi động cùng hệ thống
\end{lstlisting}

\subsection{Cài đặt MySQL và tạo cơ sở dữ liệu}

\begin{lstlisting}[caption=Cài MySQL và khởi tạo database]
sudo apt install -y mysql-server
sudo systemctl enable --now mysql
sudo mysql_secure_installation

# Tạo DB và user (nếu dùng user riêng)
sudo mysql <<'SQL'
CREATE DATABASE aivdb CHARACTER SET utf8mb4 COLLATE utf8mb4_unicode_ci;
CREATE USER 'aivuser'@'localhost' IDENTIFIED BY 'yourStrongPass';
GRANT ALL PRIVILEGES ON aivdb.* TO 'aivuser'@'localhost';
FLUSH PRIVILEGES;
SQL
\end{lstlisting}

\subsection{Cài Adminer (tuỳ chọn) qua Nginx + PHP-FPM}

\begin{lstlisting}[caption=Cài Nginx, PHP-FPM và Adminer]
sudo apt install -y nginx php-fpm php-mysql
sudo mkdir -p /var/www/adminer && cd /var/www/adminer
sudo curl -L https://www.adminer.org/latest.php -o index.php
sudo chown -R www-data:www-data /var/www/adminer

# Tạo virtual host Nginx
sudo tee /etc/nginx/sites-available/adminer > /dev/null <<'NGINX'
server {
    listen 80;
    server_name _;
    root /var/www/adminer;
    index index.php;

    location / {
        try_files $uri /index.php?$args;
    }

    location ~ \.php$ {
        include snippets/fastcgi-php.conf;
        fastcgi_pass unix:/run/php/php-fpm.sock;
    }
}
NGINX

sudo ln -s /etc/nginx/sites-available/adminer /etc/nginx/sites-enabled/adminer
sudo nginx -t && sudo systemctl reload nginx
\end{lstlisting}

\subsection{Reverse proxy Nginx cho ứng dụng}

\begin{lstlisting}[caption=Reverse proxy tới Next.js trên cổng 3000]
sudo tee /etc/nginx/sites-available/aiv-dashboard > /dev/null <<'NGINX'
server {
    listen 80;
    server_name your-domain.example.com;  # thay bằng domain thật khi có

    location / {
        proxy_set_header Host $host;
        proxy_set_header X-Real-IP $remote_addr;
        proxy_set_header X-Forwarded-For $proxy_add_x_forwarded_for;
        proxy_set_header X-Forwarded-Proto $scheme;
        proxy_pass http://127.0.0.1:3000;
    }
}
NGINX

sudo ln -s /etc/nginx/sites-available/aiv-dashboard /etc/nginx/sites-enabled/aiv-dashboard
sudo nginx -t && sudo systemctl reload nginx
\end{lstlisting}

\subsection{SSL/TLS (tuỳ chọn theo domain)}

SSL áp dụng cho \textbf{domain}, không áp dụng cho địa chỉ IP. Khi đã trỏ A record của domain về IP VPS, có thể dùng Certbot để phát hành chứng chỉ Let's Encrypt:

\begin{lstlisting}[caption=Cài Certbot và cấp chứng chỉ]
sudo apt install -y certbot python3-certbot-nginx
sudo certbot --nginx -d your-domain.example.com

# Gia hạn tự động (cron mặc định đã tạo bởi certbot)
sudo systemctl status certbot.timer
\end{lstlisting}

\subsection{Kiểm tra và vận hành}

\begin{lstlisting}[caption=Các lệnh vận hành thường dùng]
# Ứng dụng
pm2 status
pm2 logs aiv-dashboard --lines 100
pm2 restart aiv-dashboard

# Dịch vụ hệ thống
sudo systemctl status nginx
sudo systemctl status mysql

# Kết nối HTTP
curl -I http://127.0.0.1:3000
\end{lstlisting}

\subsection{Sự cố phổ biến và cách khắc phục}

\begin{itemize}
  \item \textbf{Port 3000 bận (EADDRINUSE)}: tìm và dừng tiến trình: \texttt{sudo lsof -i :3000}, \texttt{sudo kill -9 <PID>} và \texttt{pm2 restart}.
  \item \textbf{Không kết nối MySQL}: kiểm tra biến môi trường \texttt{DATABASE\_*}, thử \texttt{mysql -u root -p}, kiểm tra \texttt{bind-address} trong \texttt{/etc/mysql/my.cnf} nếu cần truy cập từ xa.
  \item \textbf{SSH Permission denied (publickey)}: kiểm tra quyền \texttt{.ssh} (700) và \texttt{authorized\_keys} (600), đúng user, đúng key.
  \item \textbf{Certbot lỗi DNS NXDOMAIN}: đảm bảo A record của domain trỏ về IP VPS trước khi cấp chứng chỉ.
\end{itemize}

\subsection{Trỏ subdomain từ Vinahost để gọi API backend}

Phần này hướng dẫn trỏ một \textbf{subdomain} (ví dụ: \texttt{api.example.vn}) từ Vinahost về VPS để sử dụng làm endpoint API backend.

\subsubsection{Bước 1: Tạo DNS record tại Vinahost}

\begin{itemize}
  \item Đăng nhập trang quản trị DNS của Vinahost.
  \item Tạo \textbf{A record} cho subdomain trỏ về IP VPS:
\end{itemize}

\begin{lstlisting}[caption=Tham chiếu cấu hình DNS (Vinahost), language=bash]
Type:   A
Host:   api            # nghĩa là api.example.vn
Value:  103.110.85.200 # IP VPS của bạn
TTL:    300            # 5 phút (tuỳ chọn)
\end{lstlisting}

Lưu ý: DNS propagation có thể mất vài phút đến 1 giờ. Kiểm tra bằng:

\begin{lstlisting}[caption=Kiểm tra phân giải DNS]
dig +short A api.example.vn
\end{lstlisting}

\subsubsection{Bước 2: Cập nhật Nginx \texttt{server\_name} cho API}

Nếu API backend chia sẻ cùng process Next.js, có thể dùng chung reverse proxy. Nếu tách route API dưới \texttt{/api}, chỉ cần thêm \texttt{server\_name} vào virtual host hiện có:

\begin{lstlisting}[caption=Thêm server_name cho subdomain API]
sudo tee /etc/nginx/sites-available/aiv-dashboard > /dev/null <<'NGINX'
server {
    listen 80;
    server_name api.example.vn;  # thêm subdomain API

    location / {
        proxy_set_header Host $host;
        proxy_set_header X-Real-IP $remote_addr;
        proxy_set_header X-Forwarded-For $proxy_add_x_forwarded_for;
        proxy_set_header X-Forwarded-Proto $scheme;
        proxy_pass http://127.0.0.1:3000;
    }
}
NGINX

sudo nginx -t && sudo systemctl reload nginx
\end{lstlisting}

\subsubsection{Bước 3: Cấp SSL (khuyến nghị)}

\begin{lstlisting}[caption=Cấp chứng chỉ Let's Encrypt cho subdomain API]
sudo apt install -y certbot python3-certbot-nginx
sudo certbot --nginx -d api.example.vn
\end{lstlisting}

Sau khi cài SSL, Nginx sẽ tự redirect HTTP→HTTPS (nếu chọn). Xác minh:

\begin{lstlisting}
curl -I https://api.example.vn/api/health
\end{lstlisting}

\subsubsection{Bước 4: Cập nhật biến môi trường và CORS}

Nếu frontend hoặc client khác gọi API qua subdomain mới, cập nhật biến môi trường và allowlist CORS:

\begin{lstlisting}[caption=Cập nhật .env trên server]
BACKEND_HOST=https://api.example.vn
NEXT_PUBLIC_API_BASE=https://api.example.vn
ALLOWED_ORIGINS=https://www.example.vn,https://app.example.vn,https://api.example.vn,http://localhost:3000
\end{lstlisting}

Ví dụ cập nhật allowlist trong \texttt{src/middleware.ts} hoặc các route API public:

\begin{lstlisting}[language=diff,caption=Bổ sung origin mới]
@@
 const ALLOWED = new Set([
   "https://www.aiesec.vn",
   "https://aiv-dashboard-ten.vercel.app",
   "https://api.example.vn",   // thêm subdomain API
   "http://localhost:3000",
 ]);
\end{lstlisting}

Khởi động lại ứng dụng:

\begin{lstlisting}
pm2 restart aiv-dashboard --update-env
\end{lstlisting}

\subsubsection{Bước 5: Kiểm thử end-to-end}

\begin{lstlisting}[caption=Kiểm thử]
# DNS
dig +short A api.example.vn

# HTTP/HTTPS
curl -I https://api.example.vn
curl -I https://api.example.vn/api/health

# Từ frontend (browser devtools):
// fetch('https://api.example.vn/api/health').then(r => r.status)
\end{lstlisting}

\section{Tổng quan về CI/CD}

Continuous Integration/Continuous Deployment (CI/CD) là một phương pháp phát triển phần mềm hiện đại, cho phép tự động hóa quá trình tích hợp, kiểm thử và triển khai ứng dụng. Trong dự án AIESEC Dashboard, tôi đã triển khai một hệ thống CI/CD sử dụng GitHub Webhook để tự động deploy ứng dụng mỗi khi có thay đổi code.

\subsection{Định nghĩa và mục tiêu}

CI/CD bao gồm hai khái niệm chính:
\begin{itemize}
    \item \textbf{Continuous Integration (CI)}: Tích hợp liên tục code từ nhiều developer vào một repository chung
    \item \textbf{Continuous Deployment (CD)}: Triển khai liên tục code đã được tích hợp vào môi trường production
\end{itemize}

Mục tiêu chính của việc triển khai CI/CD:
\begin{itemize}
    \item Tự động hóa quy trình deploy
    \item Giảm thiểu lỗi human error
    \item Tăng tốc độ phát triển và triển khai
    \item Đảm bảo tính ổn định của hệ thống
\end{itemize}

\section{Kiến trúc hệ thống CI/CD}

\subsection{Mô hình hoạt động}

Hệ thống CI/CD được thiết kế theo mô hình đơn giản nhưng hiệu quả:

\begin{center}
\texttt{Developer → GitHub Repository → Webhook Server → Production Server}
\end{center}

\subsubsection{Quy trình hoạt động chi tiết}

\begin{enumerate}
    \item \textbf{Development Phase}: Developer thực hiện commit và push code lên GitHub repository
    \item \textbf{Webhook Trigger}: GitHub tự động gửi POST request đến webhook server
    \item \textbf{Webhook Processing}: Webhook server nhận request và validate
    \item \textbf{Deployment Execution}: Script deploy được thực thi tự động
    \item \textbf{Application Restart}: PM2 restart ứng dụng với version mới
\end{enumerate}

\subsection{Các thành phần chính}

\subsubsection{GitHub Webhook}
\begin{itemize}
    \item \textbf{URL}: \texttt{https://aiv-dashboard-ten.vercel.app/hooks/aiv-dashboard-deploy}
    \item \textbf{Event}: Push to main branch
    \item \textbf{Content-Type}: application/json
    \item \textbf{Security}: IP whitelist cho GitHub servers
\end{itemize}

\subsubsection{Webhook Service}
\begin{itemize}
    \item \textbf{Port}: 9000
    \item \textbf{Configuration}: \texttt{/etc/webhook/hooks.json}
    \item \textbf{Service Management}: systemd service
    \item \textbf{Auto-restart}: Enabled với restart policy
\end{itemize}

\subsubsection{Deployment Script}
\begin{itemize}
    \item \textbf{Path}: \texttt{/root/apps/aiv-dashboard/deploy.sh}
    \item \textbf{Functionality}: Pull code, install dependencies, build, restart
    \item \textbf{Working Directory}: \texttt{/root/apps/aiv-dashboard}
\end{itemize}

\section{Triển khai và cấu hình}

\subsection{Cài đặt Webhook Service}

\begin{lstlisting}[caption=Cài đặt webhook package]
# Cài đặt webhook package
sudo apt update
sudo apt install webhook

# Tạo thư mục cấu hình
sudo mkdir -p /etc/webhook
\end{lstlisting}

\subsection{Cấu hình Webhook}

File cấu hình \texttt{/etc/webhook/hooks.json}:

\begin{lstlisting}[caption=Cấu hình webhook hooks.json, language=json]
[
  {
    "id": "aiv-dashboard-deploy",
    "execute-command": "/root/apps/aiv-dashboard/deploy.sh",
    "command-working-directory": "/root/apps/aiv-dashboard",
    "response-message": "Deployment started successfully!"
  }
]
\end{lstlisting}

\subsection{Script Deployment}

Script \texttt{/root/apps/aiv-dashboard/deploy.sh}:

\begin{lstlisting}[caption=Script deployment tự động]
#!/bin/bash

echo "🚀 Starting deployment at $(date)"

# Pull latest code
echo "📥 Pulling latest code..."
git pull origin main

# Install dependencies
echo "📦 Installing dependencies..."
npm ci

# Build application
echo "🔨 Building application..."
npm run build

# Restart PM2
echo "🔄 Restarting application..."
pm2 restart aiv-dashboard

# Check status
echo "✅ Checking status..."
pm2 status

echo "🎉 Deployment completed at $(date)"
\end{lstlisting}

\subsection{Systemd Service Configuration}

File \texttt{/etc/systemd/system/webhook.service}:

\begin{lstlisting}[caption=Cấu hình systemd service]
[Unit]
Description=Webhook Service for Auto Deploy
After=network.target
Wants=network.target

[Service]
Type=simple
User=root
Group=root
ExecStart=/usr/bin/webhook -hooks /etc/webhook/hooks.json -verbose -port 9000
Restart=always
RestartSec=10
StandardOutput=journal
StandardError=journal

[Install]
WantedBy=multi-user.target
\end{lstlisting}

\section{Kết quả triển khai}

\subsection{Lợi ích đạt được}

\subsubsection{Tự động hóa hoàn toàn}
\begin{itemize}
    \item Loại bỏ việc deploy thủ công
    \item Giảm thiểu lỗi human error
    \item Tăng tốc độ triển khai
\end{itemize}

\subsubsection{Tính khả dụng cao}
\begin{itemize}
    \item Auto-restart khi service crash
    \item Monitoring và logging đầy đủ
    \item Rollback nhanh chóng khi cần
\end{itemize}

\subsubsection{Bảo mật}
\begin{itemize}
    \item Chỉ deploy từ GitHub repository được ủy quyền
    \item IP whitelist cho GitHub servers
    \item Audit trail đầy đủ
\end{itemize}

\subsection{Metrics và hiệu suất}

\subsubsection{Thời gian deploy}
\begin{itemize}
    \item \textbf{Trước khi triển khai CI/CD}: 5-10 phút (thủ công)
    \item \textbf{Sau khi triển khai}: 30-60 giây (tự động)
\end{itemize}

\subsubsection{Tần suất deploy}
\begin{itemize}
    \item Tăng từ 1-2 lần/tuần lên 5-10 lần/tuần
    \item Khả năng deploy hotfix ngay lập tức
\end{itemize}

\subsubsection{Uptime}
\begin{itemize}
    \item 99.9\% uptime nhờ auto-restart
    \item Zero-downtime deployment
\end{itemize}

\section{Monitoring và Troubleshooting}

\subsection{Logging System}

\begin{lstlisting}[caption=Các lệnh monitoring]
# Webhook logs
sudo journalctl -u webhook -f

# Application logs
pm2 logs aiv-dashboard

# System status
pm2 status
sudo systemctl status webhook
\end{lstlisting}

\subsection{Health Checks}

\begin{lstlisting}[caption=Các lệnh health check]
# Check webhook endpoint
curl http://localhost:9000/hooks/aiv-dashboard-deploy

# Check application health
curl http://localhost:3000/api/health

# Check system resources
htop
df -h
\end{lstlisting}

\section{Cập nhật URL hệ thống: Chuyển từ IP sang Vercel Domain}

\subsection{Bối cảnh và mục tiêu}

Trong giai đoạn đầu, ứng dụng được truy cập thông qua IP \texttt{http://103.110.85.200:3000} nhằm phục vụ mục đích kiểm thử nhanh. Sau khi ổn định, yêu cầu là chuyển toàn bộ endpoint và tracking URL về domain Vercel chính thức \href{https://aiv-dashboard-ten.vercel.app}{aiv-dashboard-ten.vercel.app} \cite{vercel-app}, đảm bảo:
\begin{itemize}
  \item CORS cho phép từ domain mới (HTTPS)
  \item Các API phát sinh tracking link dùng đúng host mới
  \item Biến môi trường và dữ liệu trong cơ sở dữ liệu được đồng bộ
  \item Có kế hoạch rollback an toàn
\end{itemize}

\subsection{Các thay đổi mã nguồn}

\subsubsection{Cập nhật CORS trong middleware}
\texttt{src/middleware.ts} – thêm Vercel domain vào allowlist, thay cho IP:
\begin{lstlisting}[language=diff,caption=Update ALLOWED origins trong middleware.ts]
@@
 const ALLOWED = new Set([
-  "https://www.aiesec.vn",
-  "http://103.110.85.200:3000",
-  "http://localhost:3000",
+  "https://www.aiesec.vn",
+  "https://aiv-dashboard-ten.vercel.app",
+  "http://localhost:3000",
 ]);
\end{lstlisting}

\subsubsection{Cập nhật host khi tạo tracking link}
\texttt{src/app/api/utm/links/route.ts} – đặt fallback host về Vercel domain thay vì IP:
\begin{lstlisting}[language=diff,caption=Update generateTrackingLink]
@@
-const baseUrl = process.env.BACKEND_HOST || process.env.NEXT_PUBLIC_APP_URL || 'http://103.110.85.200:3000';
+const baseUrl = process.env.BACKEND_HOST || process.env.NEXT_PUBLIC_APP_URL || 'https://aiv-dashboard-ten.vercel.app';
\end{lstlisting}

\subsubsection{Đồng bộ CORS ở public submit endpoint}
\texttt{src/app/api/forms/by-code/[code]/submit/route.ts} – cập nhật \texttt{ALLOWED\_ORIGINS}:
\begin{lstlisting}[language=diff,caption=Update ALLOWED_ORIGINS cho submit route]
@@
- (process.env.ALLOWED_ORIGINS || "https://www.aiesec.vn,http://103.110.85.200:3000,http://localhost:3000")
+ (process.env.ALLOWED_ORIGINS || "https://www.aiesec.vn,https://aiv-dashboard-ten.vercel.app,http://localhost:3000")
\end{lstlisting}

\subsection{Cập nhật biến môi trường}

Trên máy chủ (PM2), cập nhật file \texttt{.env} và khởi động lại ứng dụng:
\begin{lstlisting}[caption=Cập nhật .env]
ALLOWED_ORIGINS=https://www.aiesec.vn,https://aiv-dashboard-ten.vercel.app,http://localhost:3000
BACKEND_HOST=https://aiv-dashboard-ten.vercel.app
NEXT_PUBLIC_APP_URL=https://aiv-dashboard-ten.vercel.app
\end{lstlisting}

\begin{lstlisting}[caption=Restart ứng dụng]
pm2 restart aiv-dashboard
pm2 logs aiv-dashboard --lines 50
\end{lstlisting}

\subsection{Cập nhật dữ liệu trong cơ sở dữ liệu}

Chuẩn hóa các tracking URL đã tạo trước đó để trỏ về domain mới (MySQL):
\begin{lstlisting}[language=sql,caption=Chuẩn hóa tracking_link về domain Vercel]
UPDATE utm_links 
SET tracking_link = REPLACE(tracking_link, 'http://103.110.85.200:3000', 'https://aiv-dashboard-ten.vercel.app')
WHERE tracking_link LIKE '%103.110.85.200:3000%';

UPDATE utm_links 
SET tracking_link = REPLACE(tracking_link, 'http://aiesecvn.digital.com:3000', 'https://aiv-dashboard-ten.vercel.app')
WHERE tracking_link LIKE '%aiesecvn.digital.com:3000%';
\end{lstlisting}

\subsection{Kiểm thử và xác minh}

\begin{itemize}
  \item Kiểm tra API test:
\begin{lstlisting}
curl https://aiv-dashboard-ten.vercel.app/api/test
\end{lstlisting}
  \item Kiểm tra CORS từ trang Webflow/Client: forms submit thành công, không lỗi preflight
  \item Tạo UTM link mới và xác nhận \texttt{tracking\_link} dùng đúng domain HTTPS
  \item Kiểm tra log ứng dụng qua PM2 đảm bảo không có lỗi 4xx/5xx bất thường
\end{itemize}

\subsection{Rủi ro và kế hoạch rollback}

\begin{itemize}
  \item \textbf{Rủi ro CORS}: origin không nằm trong allowlist \Rightarrow lỗi preflight. Khắc phục: bổ sung domain vào \texttt{ALLOWED}/\texttt{ALLOWED\_ORIGINS}
  \item \textbf{Rủi ro data mismatch}: các \texttt{tracking\_link} cũ không được chuẩn hóa \Rightarrow redirect sai host. Khắc phục: chạy script SQL trên
  \item \textbf{Rollback}: khôi phục \texttt{.env} cũ và revert code thay đổi host về IP, \texttt{pm2 restart}
\end{itemize}

\subsection{Kết quả}

Sau thay đổi:
\begin{itemize}
  \item Tất cả request sử dụng HTTPS trên domain Vercel, tương thích với Clipboard API và các chính sách bảo mật trình duyệt
  \item CORS cấu hình đúng cho domain mới, không còn phụ thuộc IP
  \item Hệ thống tracking link đồng bộ và nhất quán, giảm rủi ro khi thay đổi hạ tầng
\end{itemize}

\begin{thebibliography}{9}
\bibitem{vercel-app} AIESEC Dashboard (Vercel). Truy cập: \href{https://aiv-dashboard-ten.vercel.app}{https://aiv-dashboard-ten.vercel.app}
\end{thebibliography}

\section{Kết luận}

Việc triển khai hệ thống CI/CD đã mang lại những lợi ích đáng kể:

\begin{enumerate}
    \item \textbf{Tăng hiệu quả phát triển}: Developer có thể focus vào coding thay vì deploy
    \item \textbf{Giảm thiểu rủi ro}: Tự động hóa giảm lỗi human error
    \item \textbf{Tăng tính ổn định}: Auto-restart và monitoring đảm bảo service luôn available
    \item \textbf{Cải thiện trải nghiệm người dùng}: Deploy nhanh hơn, ít downtime
\end{enumerate}

Hệ thống CI/CD này đã chứng minh tính hiệu quả trong việc quản lý và triển khai ứng dụng AIESEC Dashboard, tạo nền tảng vững chắc cho việc phát triển và mở rộng hệ thống trong tương lai.

\subsection{Khuyến nghị phát triển}

Để cải thiện thêm hệ thống CI/CD trong tương lai:

\begin{itemize}
    \item Triển khai automated testing trước khi deploy
    \item Thêm notification system (Slack, Email) khi deploy
    \item Implement blue-green deployment để giảm downtime
    \item Thêm database migration automation
    \item Setup monitoring và alerting system
\end{itemize}

\end{document}
